\documentclass{article}
\usepackage[utf8]{inputenc}
\usepackage{enumitem}
\usepackage{amssymb}
\usepackage{amsmath}
\usepackage{array}
\usepackage{amsthm}
\usepackage[noend]{algpseudocode}
\usepackage{listings}

\setlist[enumerate]{label=(\alph*)}
\title{Pset 4}

\begin{document}

\newcommand{\Not}{\textbf{not}}
\newcommand{\AAnd}{\textbf{and}}
\newcommand{\True}{\texttt{True}}
\newcommand{\False}{\texttt{False}}

\date{March 8, 2022 }
\author{Darwin Do}

\maketitle

\begin{enumerate}
    \item Darwin Do
    \item 919941748
    \item Collaborators: 
    \item I have followed the academic integrity and collaboration policy
    \item Hours: 
\end{enumerate}

\newpage

\section{Finding Element}

\begin{enumerate}
    \item 
        We can solve this problem using binary search.
        We split our input of size $n$ into 2 groups of size $n/2$ with a "partition" element dividing between the two groups.
        If the partition element equals the target, we return the index of the partition.
        If the partition is less than the target, we recurse on the right group. Otherwise, the partition is greater than the target and we recurse on the left group.

        This algorithm works as it relies on the fact that successive elements can only change by $1$ and $p \leq t \leq q$.
        Let's define the partition element as $x$. If $x \leq t$, then $x \leq q$ and $t$ has to appear somewhere in between $x$ and $q$ as every value between $x$ and $q$ will be reached as we increase by 1 to get to $q$.
        The same logic applies for the $x \geq t$ case, $t$ would have to appear between $p$ and $x$.   
    \item 
        $\star$ Note that I'm using array inclusive array range notation 
        (i.e. $A[1:x]$) means all elements in the array $A$ from the first element to element $x$, including element $x$.
        \begin{algorithmic}
            \Function{FindElement}{$A, t$}
                \State $n = \Call{length}{A}$
                \State $x = A[\lceil \frac{n}{2} \rceil]$
                \If{$x == t$}
                    \Return{$\frac{n}{2}$}
                \ElsIf{$x > t$}
                    \Return{\Call{FindElement}{$A[1 : \frac{n}{2} - 1], t$}}
                \Else

                    \Return{\Call{FindElement}{$A[\frac{n}{2} + 1 : n], t$}}
                \EndIf
            \EndFunction
        \end{algorithmic}

    \item $T(n) = T(\frac{n}{2}) + 1$
    \item 
        \begin{align*}
            A = 1, B = 2, D = 0 \\
            R = \frac{A}{B^D} = \frac{1}{2^0} = 1
        \end{align*}
        So according to the Master Theorem: $T(n) = O(n^D\log n) = O(\log n)$
    
\end{enumerate}

\newpage
\section{Maximum Sum}
\begin{enumerate}
    \item We use a divide and conquer algorithm that divides each input into $2$ parts with size $n/2$.
            The algorithm keeps doing this until we reach the base case where $n=1$. In this case, we simply return the value of the element.
            Otherwise, for the "conquering" step of the algorithm, we let $L$ equal the solution for the left part and $R$ equal the solution for the right part as returned by our recursive calls.
            We also define $LS$ as the largest sum of the left part starting from $A[\lceil n/2\rceil]$ to $A[0]$ and
            $RS$ as the largest sum of the right part starting from $A[\lceil n/2\rceil + 1]$ to $A[n]$.
            We then return the maximum of the following options, $L$, $R$, or $LS + RS$.

            This algorithm works as each recursive call will always return the biggest contiguous sum within its input, whether that be the sum from the left side, the sum from the right side,
            or a sum that spans between the two sides.

    \item 
        \begin{algorithmic}
            \Function{MaxSum}{$A$}
                \State $n = \Call{length}{A}$

                \If{$n == 1$}
                    \Return{$A[1]$}
                \EndIf
                \State $M = \lceil \frac{n}{2} \rceil$
                \State $L = \Call{MaxSum}{A[1 : M]}$
                \State $R = \Call{MaxSum}{A[M+1 : n]}$
                \State $LS = 0; RS = 0;$
                \State $tempTotal = 0$
                \For{$i$ from $M$ to $1$}
                    \State $tempTotal = tempTotal + A[i]$
                    \If{$tempTotal > LS$}
                        \State $LS = tempTotal$
                    \EndIf
                \EndFor

                \State $tempTotal = 0$
                \For{$i$ from $M+1$ to $n$}
                \State $tempTotal = tempTotal + A[i]$
                \If{$tempTotal > RS$}
                    \State $RS = tempTotal$
                \EndIf
            \EndFor

            \Return{\Call{max}{$L, R, LS + RS$}}
            \EndFunction
        \end{algorithmic}
    \item $T(n) = 2T(\frac{n}{2}) + O(n)$
    \item 
        \begin{align*}
            A = 2, B = 2, D = 1 \\
            R = \frac{A}{B^D} = \frac{2}{2^1} = 1
        \end{align*}
        So according to the Master Theorem: $T(n) = O(n^D\log n) = O(n\log n)$
\end{enumerate}

\newpage
\section{Longest Common Substring}
\begin{enumerate}
    \item $T(i, j)$ is the length of the largest common substring that ends at index $i$ in $x$ and index $j$ in $y$.
            The final answer of the algorithm will be the max value of all table entries.
    \item 
        \begin{align*}
            T(i, j) = 
            \begin{array}{cc}
                \Bigg\{ & 
                \begin{array}{cc}
                    T(i-1, j-1) + 1 & x[i] = y[j] \\
                    0 & x[i] \neq y[j] 
                \end{array}
            \end{array}
        \end{align*}
        The base cases are the indexes of the recurrence before the first character. In this homework with 1-indexing, our base cases would be $T(0, j) = 0$ and $T(i, 0) = 0$.
        
        The recurrence works as each value in the table holds the value of the longest substring that ends at index $i$ in string $x$ and index $j$ in string $y$. If this common substring with length $L$ can grow,
        then $x[i+1] = y[j+1]$. We know then that the current common substring now has length $L' = L+1$. Otherwise, if the next characters are not equal, then the common substring has ended and the length
        of the common substring \emph{that ends at indexes \textbf{i} and \textbf{j} in strings \textbf{x} and \textbf{y} respectively} is $0$. The piecewise function of the recurrence accounts for this.
    \item 
        \begin{algorithmic}
            \Function{LongestCommonSubstring}{$x, y$}
                \ForAll{$i$ from $0$ to $n$ and $j$ from $0$ to $m$}
                    \State $sub[0][j] = 0; \text{ } sub[i][0] = 0$
                \EndFor

                \State $maxLength = 0$
                \ForAll{$i$ from $1$ to $n$}
                    \ForAll{$j$ from $1$ to $m$}
                        \If{$x[i] == y[j]$}
                            \State $sub[i][j] = sub[i-1][j-1] + 1$
                            \If{$sub[i][j] > maxLength$}
                                $maxLength = sub[i][j]$
                            \EndIf
                        \Else
                            \State $sub[i][j] = 0$
                        \EndIf
                    \EndFor
                \EndFor
                \Return{$maxLength$}
            \EndFunction
        \end{algorithmic}
    \item The runtime is $O(nm)$. We iterate through all $n$ characters in string $x$ and for every character, we look at every character in string $y$ of length $m$.
            There is a constant amount of work for each inner iteration as we are simply updating an array with a previously supplied value. Therefore the runtime is $O(nm)$ total.

\end{enumerate}

\newpage
\section{Infinitely Many Rods}
\begin{enumerate}
    \item $T(i) = \True$ if a rod of length $i$ can be assembled with the given subrods and $T(i) = \False$ anyways.
            The final answer of the algorithm will be the value stored at $T(i)$
    \item         
        \begin{align*}
            T(i) = 
            \begin{array}{cc}
                \Bigg\{ & 
                \begin{array}{cc}
                    True & \forall (1 \leq x \leq n)(\exists \ell_x: i-\ell_x > 0 \land T(i-\ell_x) = \True)   \\
                    False & \forall (1 \leq x \leq n)(\nexists \ell_x: i-\ell_x > 0 \land T(i-\ell_x) = \True)
                \end{array}
            \end{array}
        \end{align*}

        The base case is $T(0) = \True$

        The recurrence is correct as at index $i$ with all the recurrences below $i$ already solved, we know that we only need to add a minimum of one rod from one of the previously solved solutions to form a rod of length $i$ if it exists.
        Therefore we cycle through all the availble rod sizes $\ell_x$ and see if the recurrence at $i - \ell_x$ is true. If it is, then we know we are able to add a rod of size $\ell_x$ to form a rod of size $i$ and $T(i)$ is true.
        Otherwise if none of the supplied rods work, we know that it is not possible to form a rod of size $i$ with our given inventory. 

        The base case is $T(0) = \True$ as we want singleton rods to be true (i.e. a rod of size 3 can be formed by a single rod of length 3). 
        This also intuitively makes sense as it doesn't matter what length of rods we have in order to make "no" rod.

    \item 
        \begin{algorithmic}
            \Function{CheckValidRod}{$\ell_1...\ell_n, S$}
                \State $rod[0] = \True$
                \ForAll{$s$ from $1$ to $S$}
                    \ForAll{$i$ from $1$ to $n$}
                        \If{$s - \ell_i >= 0 \: \AAnd \: rod[s - \ell_i]$}
                            \State $rod[s] = \True$
                            \State \textbf{break}
                        \Else
                            \State $rod[s] = \False$
                        \EndIf
                    \EndFor
                \EndFor

                \Return{$rod[S]$}
            \EndFunction        
        \end{algorithmic}

    \item The runtime is $O(nS)$. In our dynamic programming algorithm, we update our table of solutions from $rod[1]$ to $rod[S]$.
            Each table update takes $O(n)$ time as the worst case scenario is to cycle through all rods supplied. Therefore the total runtime is $O(nS)$
\end{enumerate}

\newpage
\section{At Most One Rod}
\begin{enumerate}
    \item 
\end{enumerate}
\end{document}
