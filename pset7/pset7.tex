\documentclass{article}
\usepackage[utf8]{inputenc}
\usepackage{enumitem}
\usepackage{amssymb}
\usepackage{amsmath}
\usepackage{array}
\usepackage{amsthm}
\usepackage[noend]{algpseudocode}
\usepackage{listings}

\setlist[enumerate]{label=(\alph*)}
\title{Pset 7}

\begin{document}

\newcommand{\Not}{\textbf{not}}
\newcommand{\AAnd}{\textbf{and}}
\newcommand{\Or}{\textbf{or}}
\newcommand{\True}{\texttt{True}}
\newcommand{\False}{\texttt{False}}

\date{April 26, 2022 }
\author{Darwin Do}

\maketitle

\begin{enumerate}
    \item Darwin Do
    \item 919941748
    \item Collaborators: Graham Stodolski, Raffael Davila
    \item I have followed the academic integrity and collaboration policy
    \item Hours: 5
\end{enumerate}

\newpage

\section{Strongly Independent Set}

To prove that \texttt{SIS} is NP-hard, we show that $\texttt{IS} \leq_p \texttt{SIS}$ where \texttt{IS} is the \texttt{Independent-Set} problem as discussed in lecture.

\begin{proof}
    Let $G = (V, E)$ be an undirected graph that is a valid input to the \texttt{IS} problem.
    We will form $G' = (V', E')$ by replacing every edge $(u, v) \in E$ with two edges, $(u, w)$ and $(w, v)$ where $w$ is a new node.
    Then, we connect all the $w$ nodes together.

    If $G$ is \texttt{yes} on the \texttt{IS} problem, then $G'$ will also be \texttt{yes} on the \texttt{SIS} problem with the same $k$.
    This is because any nodes in the independent set $S$ of size $\geq k$ will have a distance between each other of at least 2 by definition. 
    Due to the adding of the $(u, w)$ and $(w, v)$ edges, this distance will become at least 4. So the nodes in $S$ are a valid strong independent set in $G'$ with size $\geq k$.
    Otherwise, if the candidate set $S$ is \texttt{no} for the \texttt{IS} problem on $G$, $G'$ does not have an independent set of size $\geq k$.
    There will be at least 2 vertices $u,v \in S$ that are connected and have a distance of 1 between each other. When we create the $(u, w)$ $(w, v)$ edge pairs in $G'$, the distance between $u$ and $v$ in $G'$ will be 2 and break the \texttt{SIS} principle.

    If $G'$ is \texttt{yes} on the \texttt{SIS} problem, then $G$ will also be \texttt{yes} on the \texttt{IS} problem with the same $k$.
    Let $S$ be the strong independent set of size $k$ in $G'$. 
    We know that there can be no $w$ nodes in $S$ as all the $w$ nodes are connected to each other so you can reach any node from a $w$ node within 2 edges.
    That means we can remove the $w$ nodes from $G'$ and re-connect the $(u, v)$ nodes to get $G$. The nodes in the strong independent set $S$ are the same nodes in the independent set for $G$ since none of the vertices in $S$ have a distance of 1 between them by definition of \texttt{SIS}.
    Otherwise, if the candidate set $S$ is \texttt{no} for the \texttt{SIS} problem on $G'$, $G$ does not have an independent set of size $\geq k$. 
    In this case, is at least one pair of nodes $u, v \in S$ whose distance is $\leq 2$. If it is 1, then we have already violated the independent set principle.
    If the distance is 2, then we know the path from $u$ to $v$ is $(u, w) \rightarrow (w, v)$ by construction. Once we remove the $w$ node and reconnect the $(u, v)$, they will be neighbors and violate the independent set principle.

    We have shown $\texttt{IS} \leq_p \texttt{SIS}$. Since \texttt{IS} is NP-hard, \texttt{SIS} is NP-hard as well.
\end{proof}

Now we show that $\texttt{SIS} \in NP$.

\begin{proof}
    Let $S$ be a candidate strongly independent set for graph $G$.
    For all $v \in S$, we perform a BFS from $v$ that only goes 2 levels from $v$. 
    If we encounter any other nodes in $S$ from our 2-level BFS, we know that this candidate is not a strongly independent set.
    Otherwise if complete this process for all $v \in S$ and no elements of $S$ are in the 2-level BFS, we know this is an strongly independent set.

    This modified version of BFS is faster than BFS as it only goes 2 levels. The runtime of normal BFS is $O(|V| + |E|)$.
    We are running this modified version of BFS on every node in $S$, so the upper bound runtime is $O( (|V| + |E|)\cdot |V| )$ which is polynomial to the graph.
    Verifying a candidate takes a polynomial amount of time so \texttt{SIS} is in NP.
\end{proof}

\texttt{SIS} is in NP and is NP-hard so it is NP-complete.

\newpage
\section{Dominating Set}
We first show that \texttt{DS} is NP-hard by showing that $\texttt{VC} \leq_p \texttt{DS}$ where \texttt{VC} is the \texttt{Vertex-Cover} problem as discussed in lecture.
\begin{proof}
    Let $G = (V, E)$ be an undirected graph that is a valid input to the \texttt{VC} problem.
    We will form $G' = (V', E')$ by replacing every edge $(u, v) \in E$ with two edges, $(u, w)$ and $(w, v)$ where $w$ is a new node.
    Then we connect all the vertices $v \in V$ together with edges.

    If $G$ is \texttt{yes} on the \texttt{VC} problem then there is a vertex set $S$ with size $\leq k$ where every edge has at least one vertex in $S$.
    This set $S$ will also be a valid dominating set in $G'$. We know that every edge in $G$ is touching at least one node in $S$.
    By construction, that means every $w$ node will be touching a node in $S$. Then since we connect all the nodes $v \in V$ together, we know all $v \in V$ is either a node in $S$ or touching a $S$ node.
    Every $v \in V'$ is either in $S$ or touching a node in $S$ so $G'$ is a dominating set of size $\leq k$.
    Otherwise, if set $S$ is \texttt{no} for \texttt{VC} on $G$, there will be at least one edge $(u, v) \in E$ where $u$ and $v$ are not in $S$.
    In $G'$, the $w$ node connecting to the $u, v$ nodes will be connected to 2 nodes not in $S$ breaking the dominating set principle.

    If $G'$ is \texttt{yes} on the \texttt{DS} problem, then $G$ is also \texttt{yes} on the \texttt{VC} problem.
    Let $S$ be the dominating set in $G'$ of size $\leq k$. By definition, we know that every $w$ node is a neighbor of at least 1 node in $S$.

    When we remove the $w$ node and the $(v, w)$, $(w, u)$ edges, we connect $(v, u)$ again to form the edge in $E$. 
    Since we know that at least $u$ or $v$ is in $S$, the $(v, u)$ edge follows the vertex-cover principle. 
    We do this step for all edges in in $G'$ that are connected to a $w$ node. For edges that are not connected to a $w$ node, (instead connected to each $v \in V$), we 
    simply remove these from $G'$ to get $G$ so we don't need to worry about them. Therefore the $S$ set is a valid vertex cover of size $\leq k$.
    Otherwise, if set $S$ is \texttt{no} for \texttt{DS} on $G'$, there will be at least one node $v^*$ who has 0 neighbors in $S$.
    We know $v^* \notin V$ as every $v \in V$ is connected to each other so all $v \in V$ is either in $S$ or a neighbor of $S$ since $|S| > 0$.
    That means $v^*$ is a $w$ node with edges $(u, w)$ and $(w, v)$ where $v$ and $u$ are not in $S$. 
    When deleting the $w$ nodes to form $G$, the $(u, v)$ edge will violate the vertex-cover principle and $S$ will not be a valid vertex-cover for $G$.
    
\end{proof}

Now we show $\texttt{DS} \in NP$

\begin{proof}
    Let $S$ be a candidate dominating set for graph $G$ with size $\leq k$.
    For all $v \in V$, we check if either $v$ is in $S$ or if one of the neighbors of $v$ is in $S$.
    If this condition fails, we know that $S$ is not a valid dominating set.
    Otherwise if none of the nodes fail this check, we know $S$ is a valid dominating set of size $\leq k$ on $G$.
    
    The worst-case runtime for this is $O(|V|^2)$ as we iterate through all the nodes and look at each node's neighbors.
    This is polynomial time so \texttt{DS} is in NP.
\end{proof}

\texttt{DS} is in NP and is NP-hard so it is NP-complete.

\end{document}
