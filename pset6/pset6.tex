\documentclass{article}
\usepackage[utf8]{inputenc}
\usepackage{enumitem}
\usepackage{amssymb}
\usepackage{amsmath}
\usepackage{array}
\usepackage{amsthm}
\usepackage[noend]{algpseudocode}
\usepackage{listings}

\setlist[enumerate]{label=(\alph*)}
\title{Pset 6}

\begin{document}

\newcommand{\Not}{\textbf{not}}
\newcommand{\AAnd}{\textbf{and}}
\newcommand{\Or}{\textbf{or}}
\newcommand{\True}{\texttt{True} }
\newcommand{\False}{\texttt{False} }

\date{April 17, 2022 }
\author{Darwin Do}

\maketitle

\begin{enumerate}
    \item Darwin Do
    \item 919941748
    \item Collaborators: 
    \item I have followed the academic integrity and collaboration policy
    \item Hours: 
\end{enumerate}

\newpage

\section{Minimum Spanning Tree}
\begin{enumerate}
    \item Starting from a random vertex $v$ in $S$, use DFS to \texttt{Explore} and count the number of reachable nodes from $v$.
            If the number of nodes in this component-from-$v$ is not equal to the total number of vertices in $S$, we know $S$ is not connected.
            Otherwise, all vertices in $S$ are reachable from $v$ and $S$ is connected. This takes $O(|E| + |V|)$ time through DFS.

    \item Starting from a random vertex $v$ in $S$, use DFS to \texttt{Explore} $S$ and keep track of the nodes visited. 
            If there is a back-edge detected where a node is a neighbor to an already-visited node that isn't the node used to get to that current node, we know $S$ contains a cycle.
            Otherwise, if we manage to DFS through all of $S$ and detect no back-edge, $S$ should be cycle-free. This takes $O(|E| + |V|)$ time through DFS.
            
    \item Starting from a random vertex $v$ in $S$, use DFS to \texttt{Explore} all of $S$. If we fail to encounter a node in $S$ that exists in $V$, we know that $S$ does not have vertex set $V$.
            Otherwise if all the nodes in $V$ are present in $S$, we have vertex set $V$. This takes $O(|E| + |V|)$ time through DFS.

    \item First we check that $S$ is connected, cycle-free, and has vertex set $V$ from the steps outlined in a - c. If any of these conditions fail then $S$ is not a spanning tree.
    We run Prim's algorithm on $G$ to get $G' = (V', E')$, the MST of $G$. We DFS on the $G'$ and sum all the edge weights to $w_{g'}$. We then DFS on $S$ and sum the edge weights to $w_{s}$.
    If $w_s > w_{g'}$ then $S$ is not a MST, otherwise if $w_s = w_{g'}$, $S$ is a MST. We run Prim's once and DFS twice. The runtime of Prim is $O((|V| + |E|)\log |V|)$ which dominates the linear runtime of DFS/BFS.
\end{enumerate}

The dominating runtime of all the substeps to check if $S$ is a MST is $O((|V| + |E|)\log |V|)$.
That means we can verify whether a candidate is a MST in polynomial time, so the \texttt{MST} problem is in \textbf{NP}.

\newpage
\section{SAT Variations}
\begin{enumerate}
        \item Since each clause has to be full of either all \True or all \False terms, we can exploit this and turn all 3-term clauses into 2-term clauses.
                Let $c = (t_a \lor t_b \lor t_c )$ be a generic clause in this CNF. We want either the terms to be either all \True or \False:
                \begin{align*}
                        (t_a \land t_b \land t_c) \lor (\bar{t}_a \land \bar{t}_b \land \bar{t}_c)
                \end{align*}

                We can distribute this and simplify to get the following expression:
                \begin{align*}
                        (t_a \lor \bar{t}_b) \land (t_a \lor \bar{t}_c) \land (t_b \lor \bar{t}_a)
                        \land (t_b \lor \bar{t}_c) \land (t_c \lor \bar{t}_a) \land (t_c \lor \bar{t}_b) 
                \end{align*}                
                
                We can do this process for every clause in the CNF to obtain a \texttt{2SAT} problem. 
                We have reduced \texttt{Prob1} to \texttt{2SAT}. We know that \texttt{2SAT} is solvable in polynomial time through the technique discussed in lecture of creating a node for each variable and its negation, and adding edges between vertices that are constrained.
                We then check the graph to see if a variable is neighbors with its negated edge. If so, we know there is no assignment possible. 

        \newpage
        \item Let $f$ be a valid input to the \texttt{3SAT} problem. By construction, $f$ already has 3 literals in each clause.
                We just need to ensure that no variable appears in more than 3 clauses. Let $A$ be the set of variables that appear in more than 3 clauses.
                For every $x \in A$ in $n$ different clauses, we replace every $x$ with a separate set of variables $y_{1} ... y_n$ and ensure that the $y_1 ... y_n$ variables have the same value.
                This is a biconditional around all of the $y$ variables. i.e: $y_{1} \implies y_2 \implies ... \implies y_n \implies y_1$. 
                An implication of $y_1 \implies y_2$ is logically equivalent to $\bar{y}_1 \lor y_2$. That means we can append $(\bar{y}_1 \lor y_2) \land (\bar{y}_2 \lor y_3) \land ... \land (\bar{y}_n \lor y_1)$ to the CNF to ensure this conditional property.
                The ensurance of the property only uses 2 instances of each $y_i$, leaving the third $y_i$ to replace the $x \in A$. Therefore no $y_i$ will exceed three clauses.

                At this point we have an instance of \texttt{Prob2} as each clause has at most 3 literals and each literal appears in at most 3 clauses.
                
                % TODO: do the biconditonal proof here

                This transformation takes polynomial time as we spend a linear amount of time in terms of literals to create set $A$ by iterating through all the literals.
                Then for each $x \in A$, we spend a linear amount of creating the $y$ variables as we know the number of $y$ variables will always be less than the total number of literals. 
                Together, this makes for a polynomial runtime to do this transformation.

                We have reduced from \texttt{3SAT} to \texttt{Prob2} so $\texttt{3SAT} \leq_p \texttt{Prob2}$. 
                Since $\texttt{3SAT}$ is NP-hard as discussed in lecture, we know that \texttt{Prob2} is NP-hard as well.
                
        \newpage
        \item Let $f$ be a valid input to the \texttt{SAT} problem (a boolean formula in CNF).
                 Let $f'$ be another CNF transformed from $f$ to be of the form $f' = f \land C$ where $C$ is a clause with the following unused-in-$f$ literals: $(x_1 \lor x_2)$.

                This is a valid reduction transformation. Say that $f$ has a satisfying assignment so \texttt{SAT} is \texttt{Yes}. That means that $f'$ is also \texttt{Yes} for \texttt{Prob3}
                as we can make the following assignments to create 3 distinct satisfying assignments:
                \begin{enumerate}
                        \item The assignments from $f$ and $(x_1 = 1, x_2 = 0)$
                        \item The assignments from $f$ and $(x_1 = 0, x_2 = 1)$
                        \item The assignments from $f$ and $(x_1 = 1, x_2 = 1)$
                \end{enumerate}

                Say that $f$ does not have a satisfying assignment so \texttt{SAT} is \texttt{No}. 
                Since there is no valid assignment for $f$, there can be no satisfying assignment for $f'$ either since $f'$ is composed of $f$ appended with an independant part.
                
                Say that $f'$ has at least 3 satisfying assignments. That means
                
                \begin{align*}
                        f' = f \land C = \True
                \end{align*} 
                so $f$ has to be \True and have a satisfying assignment as well. 
                Likewise, if $f'$ does not have at least 3 satisfying assignments, then $f$ has to be \False and not have any satisfiable assignments as otherwise, the $C$ clause will always add 3 combinations of valid assignments.

                The creation of $f'$ is polynomial as we just need to add a new clause with 2 literals which takes constant time.

                This is a valid reduction from \texttt{SAT} to \texttt{Prob3} so $\texttt{SAT} \leq_p \texttt{Prob3}$.
                Since $\texttt{SAT}$ is NP-hard as discussed in lecture, we know that \texttt{Prob3} is NP-hard as well.
\end{enumerate}
\newpage
\section{Graph Coloring}
\subsection{In NP}
First we show that \texttt{4-COLORING} is in \textbf{NP}.
Say we have graph $G = (V, E)$ with $k=4$ colors and a $c(v)$ color assignment. We can itertate through all the nodes in $G$ with DFS and check if any node has edges with the same color.
If a node does, we know that this assignment $c(v)$ is not a valid \texttt{4-COLORING} assignment. Otherwise if we iterate through all the nodes and see that no node has two edges of the same color,
the $c(v)$ candidate is a valid \texttt{4-COLORING} assignment.

This operation just uses DFS to explore the whole graph which is linear time. Therefore \texttt{4-COLORING} is in NP as this is also in polynomial time.

\subsection{NP-Hard}
Now we show \texttt{4-COLORING} is NP-hard.
Say we have a graph $G = (V, E)$ with a \texttt{3-COLORING} input form of $k=3$ colors and a $c(v)$ color assignment.

We can reduce this instance to an instance of \texttt{4-COLORING} with graph $G' = (V', E')$.
We form $G'$ by adding a new node $v'$ with color $c(v') = 4$ to $V$ and an edge $(v, v') \in E'$ to any $v \in V$.

The only difference between $G$ and $G'$ is this vertex $v'$ with color $c(v') = 4$. 
Therefore if $G$ is a valid \texttt{3-COLORING}, $G'$ is a valid \texttt{4-COLORING} as no node $v \in V$ will break the color-neighbor property through the definition of \texttt{3-COLORING}.
No neighbor of $v'$ will break this property either as no other node will have color 4.
Likewise, if $G$ is not a valid \texttt{3-COLORING} then $G'$ is not a valid \texttt{4-COLORING} as somewhere in $V$ the color principle is already broken.

Assume that $G'$ is a valid \texttt{4-COLORING}. By construction, we know that all nodes $v \in V'$ follow the color-neighbor \texttt{COLORING} principle.
We also know that vertex $v'$ is the only node with color 4 in $G'$. 
Therefore we also know that $G$ is a valid \texttt{3-COLORING} because we can remove vertex $v'$ to get a graph with only 3 colors where all nodes follow the color-neighbor principle.
Likewise, if $G'$ is not a valid \texttt{4-COLORING}, we know there are at least 2 nodes in $G'$ that share the same color. It can't be $v'$ as no other node has color 4.
Therefore some pair of node neighbors $(u, v) \in V$ have the same color so \texttt{3-COLORING} is \texttt{No}.

Doing this transformation from $G$ to $G'$ is in polynomial time as it is just adding a single node and edge.

We have shown that $\texttt{3-COLORING} \leq_p \texttt{4-COLORING}$. Since \texttt{3-COLORING} is NP-hard, \texttt{4-COLORING} is NP-hard as well.

Therefore, \texttt{3-COLORING} is NP-complete.

\end{document}
