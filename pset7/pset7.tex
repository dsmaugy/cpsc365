\documentclass{article}
\usepackage[utf8]{inputenc}
\usepackage{enumitem}
\usepackage{amssymb}
\usepackage{amsmath}
\usepackage{array}
\usepackage{amsthm}
\usepackage[noend]{algpseudocode}
\usepackage{listings}

\setlist[enumerate]{label=(\alph*)}
\title{Pset 7}

\begin{document}

\newcommand{\Not}{\textbf{not}}
\newcommand{\AAnd}{\textbf{and}}
\newcommand{\Or}{\textbf{or}}
\newcommand{\True}{\texttt{True}}
\newcommand{\False}{\texttt{False}}

\date{April 26, 2022 }
\author{Darwin Do}

\maketitle

\begin{enumerate}
    \item Darwin Do
    \item 919941748
    \item Collaborators: Graham Stodolski, Raffael Davila
    \item I have followed the academic integrity and collaboration policy
    \item Hours: 5.5
\end{enumerate}

\newpage

\section{Strongly Independent Set}

To prove that \texttt{SIS} is NP-hard, we show that $\texttt{IS} \leq_p \texttt{SIS}$ where \texttt{IS} is the \texttt{Independent-Set} problem as discussed in lecture.

\begin{proof}
    Let $G = (V, E)$ be an undirected graph that is a valid input to the \texttt{IS} problem.
    We will form $G' = (V', E')$ from $G$ by replacing every edge $(u, v) \in E$ with two edges, $(u, w)$ and $(w, v)$ where $w$ is a new node.
    Then, we connect all the $w$ nodes together.

    If $G$ is \texttt{yes} on the \texttt{IS} problem, then $G'$ will also be \texttt{yes} on the \texttt{SIS} problem with the same $k$.
    This is because any nodes in the independent set $S$ of size $\geq k$ will have a distance between each other of at least 2 by definition. 
    Due to the adding of the $(u, w)$ and $(w, v)$ edges, this distance will become at least 4. So the nodes in $S$ are a valid strong independent set in $G'$ with size $\geq k$.
    Otherwise, if the candidate set $S$ is \texttt{no} for the \texttt{IS} problem on $G$, $G'$ does not have an independent set of size $\geq k$.
    There will be at least 2 vertices $u,v \in S$ that are connected and have a distance of 1 between each other. When we create the $(u, w)$ $(w, v)$ edge pairs in $G'$, the distance between $u$ and $v$ in $G'$ will be 2 and break the \texttt{SIS} principle.

    If $G'$ is \texttt{yes} on the \texttt{SIS} problem, then $G$ will also be \texttt{yes} on the \texttt{IS} problem with the same $k$.
    Let $S$ be the strong independent set of size $k$ in $G'$. 
    We know that there can be no $w$ nodes in $S$ as all the $w$ nodes are connected to each other so you can reach any node from a $w$ node within 2 edges.
    That means we can remove the $w$ nodes from $G'$ and re-connect the $(u, v)$ nodes to get $G$. The nodes in the strong independent set $S$ are the same nodes in the independent set for $G$ since none of the vertices in $S$ have a distance of 1 between them by definition of \texttt{SIS}.
    Otherwise, if the candidate set $S$ is \texttt{no} for the \texttt{SIS} problem on $G'$, $G$ does not have an independent set of size $\geq k$. 
    In this case, is at least one pair of nodes $u, v \in S$ whose distance is $\leq 2$. If it is 1, then we have already violated the independent set principle.
    If the distance is 2, then we know the path from $u$ to $v$ is $(u, w) \rightarrow (w, v)$ by construction. Once we remove the $w$ node and reconnect the $(u, v)$, they will be neighbors and violate the independent set principle.

    The transformation from $G \rightarrow G'$ is polynomial as for every $e \in E$ we are adding 1 extra node and up to $O(|V|)$ extra edges so the runtime is bound by $O(|E| \cdot |V|)$
    We have shown $\texttt{IS} \leq_p \texttt{SIS}$. Since \texttt{IS} is NP-hard, \texttt{SIS} is NP-hard as well.
\end{proof}

Now we show that $\texttt{SIS} \in NP$.

\begin{proof}
    Let $S$ be a candidate strongly independent set for graph $G$.
    For all $v \in S$, we perform a BFS from $v$ that only goes 2 levels from $v$. 
    If we encounter any other nodes in $S$ from our 2-level BFS, we know that this candidate is not a strongly independent set.
    Otherwise if complete this process for all $v \in S$ and no elements of $S$ are in the 2-level BFS, we know this is an strongly independent set.

    This modified version of BFS is faster than BFS as it only goes 2 levels. The runtime of normal BFS is $O(|V| + |E|)$.
    We are running this modified version of BFS on every node in $S$, so the upper bound runtime is $O( (|V| + |E|)\cdot |V| )$ which is polynomial to the graph.
    Verifying a candidate takes a polynomial amount of time so \texttt{SIS} is in NP.
\end{proof}

\texttt{SIS} is in NP and is NP-hard so it is NP-complete.

\newpage
\section{Dominating Set}
We first show that \texttt{DS} is NP-hard by showing that $\texttt{VC} \leq_p \texttt{DS}$ where \texttt{VC} is the \texttt{Vertex-Cover} problem as discussed in lecture.
\begin{proof}
    Let $G = (V, E)$ be an undirected graph that is a valid input to the \texttt{VC} problem.
    We will form $G' = (V', E')$ from $G$ by adding a $w$ node to $G$ and edges $(u, w)$ and $(w, v)$ for all $(u, v) \in E$.
    We also remove any "floating" nodes that have degree 0 in $G$ as these nodes don't affect the \texttt{VC} problem but do affect \texttt{DS}.

    If $G$ is \texttt{yes} on the \texttt{VC} problem then there is a vertex set $S$ with size $\leq k$ where every edge has at least one vertex in $S$.
    This set $S$ will also be a valid dominating set in $G'$. We know that every $(u, v)$ edge in $G$ is touching at least one node in $S$ so at least either $u$ or $v$ is in $S$.
    By construction, that means every $w$ node will be touching a node in $S$ as well.
    So every $v \in V'$ is either in $S$ or touching a node in $S$ so $G'$ is a dominating set of size $\leq k$.
    Otherwise, if set $S$ is \texttt{no} for \texttt{VC} on $G$, there will be at least one edge $(u, v) \in E$ where $u$ and $v$ are not in $S$.
    In $G'$, the $w$ node connecting to the $u, v$ nodes will be connected to 2 nodes not in $S$ breaking the dominating set principle.

    If $G'$ is \texttt{yes} on the \texttt{DS} problem, then $G$ is also \texttt{yes} on the \texttt{VC} problem.
    Let $S$ be the dominating set in $G'$ of size $\leq k$. By definition, we know that every $v \in V'$ is in $S$ or is neighbors with a node in $S$.
    If $S$ does not contain any $w$ nodes, we can simply delete the $w$ nodes and their corresponding edges to obtain $G$ and retain the Vertex-Cover principle.
    Otherwise, if a $w$ node is part of the dominating set with the following edges: $(u, w), (w, v)$, we can "shift" that role and promote either $u$ or $v$ to be in the dominating set.
    The $w$ node has no other neighbors so the "work" that this node is doing to ensure the dominating set principle will be transferred to either $u$ or $v$ with no change to the number of elements in the dominating set.
    Once all the $w$ nodes in $S$ have been reassigned to their neighbors, we have a modified set $S'$ where no $v \in S'$ is a $w$ node.
    From the earlier justification, once we delete all the $w$ nodes, $S'$ will be a vertex cover on $G$.
    Otherwise, if set $S$ is \texttt{no} for \texttt{DS} on $G'$, there will be at least one node $v^*$ who is not in $S$ and has 0 neighbors in $S$.
    If the $v^*$ node is a $w$ node, then its neighbor nodes $u, v$ are not in $S$ and the $(u, v)$ edge in $G$ breaks the vertex cover principle.
    If the $v^*$ node is not a $w$ node and is part of $V$, then any $(v^*, u)$ edge breaks the vertex cover principle.

    The transformation from $G \rightarrow G'$ is polynomial as we are only adding 1 new node and a pair of edges for every $e \in E$ which takes linear time.
    We have shown $\texttt{VC} \leq_p \texttt{DS}$. Since \texttt{VC} is NP-hard, \texttt{DS} is NP-hard as well.
\end{proof}

Now we show $\texttt{DS} \in NP$

\begin{proof}
    Let $S$ be a candidate dominating set for graph $G$ with size $\leq k$.
    For all $v \in V$, we check if either $v$ is in $S$ or if one of the neighbors of $v$ is in $S$.
    If this condition fails, we know that $S$ is not a valid dominating set.
    Otherwise if none of the nodes fail this check, we know $S$ is a valid dominating set of size $\leq k$ on $G$.
    
    The worst-case runtime for this is $O(|V|^2)$ as we iterate through all the nodes and look at each node's neighbors.
    This is polynomial time so \texttt{DS} is in NP.
\end{proof}

\texttt{DS} is in NP and is NP-hard so it is NP-complete.

\end{document}
