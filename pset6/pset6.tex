\documentclass{article}
\usepackage[utf8]{inputenc}
\usepackage{enumitem}
\usepackage{amssymb}
\usepackage{amsmath}
\usepackage{array}
\usepackage{amsthm}
\usepackage[noend]{algpseudocode}
\usepackage{listings}

\setlist[enumerate]{label=(\alph*)}
\title{Pset 6}

\begin{document}

\newcommand{\Not}{\textbf{not}}
\newcommand{\AAnd}{\textbf{and}}
\newcommand{\Or}{\textbf{or}}
\newcommand{\True}{\texttt{True}}
\newcommand{\False}{\texttt{False}}

\date{April 17, 2022 }
\author{Darwin Do}

\maketitle

\begin{enumerate}
    \item Darwin Do
    \item 919941748
    \item Collaborators: 
    \item I have followed the academic integrity and collaboration policy
    \item Hours: 
\end{enumerate}

\newpage

\section{Minimum Spanning Tree}
\begin{enumerate}
    \item Starting from a random vertex $v$ in $S$, use DFS to \texttt{Explore} and count the number of reachable nodes from $v$.
            If the number of nodes in this component-from-$v$ is not equal to the total number of vertices in $S$, we know $S$ is not connected.
            Otherwise, all vertices in $S$ are reachable from $v$ and $S$ is connected. This takes $O(|E| + |V|)$ time through DFS.

    \item Starting from a random vertex $v$ in $S$, use DFS to \texttt{Explore} $S$ and keep track of the nodes visited. 
            If there is a back-edge detected where a node is a neighbor to an already-visited node that isn't the node used to get to that current node, we know $S$ contains a cycle.
            Otherwise, if we manage to DFS through all of $S$ and detect no back-edge, $S$ should be cycle-free. This takes $O(|E| + |V|)$ time through DFS.
            
    \item Starting from a random vertex $v$ in $S$, use DFS to \texttt{Explore} all of $S$. If we fail to encounter a node in $S$ that exists in $V$, we know that $S$ does not have vertex set $V$.
            Otherwise if all the nodes in $V$ are present in $S$, we have vertex set $V$. This takes $O(|E| + |V|)$ time through DFS.

    \item Check MST
\end{enumerate}

\newpage
\section{SAT Variations}

\newpage
\section{Graph Coloring}
\end{document}
